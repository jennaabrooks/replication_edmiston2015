% Options for packages loaded elsewhere
\PassOptionsToPackage{unicode}{hyperref}
\PassOptionsToPackage{hyphens}{url}
\PassOptionsToPackage{dvipsnames,svgnames,x11names}{xcolor}
%
\documentclass[
  letterpaper,
  DIV=11,
  numbers=noendperiod]{scrartcl}

\usepackage{amsmath,amssymb}
\usepackage{iftex}
\ifPDFTeX
  \usepackage[T1]{fontenc}
  \usepackage[utf8]{inputenc}
  \usepackage{textcomp} % provide euro and other symbols
\else % if luatex or xetex
  \usepackage{unicode-math}
  \defaultfontfeatures{Scale=MatchLowercase}
  \defaultfontfeatures[\rmfamily]{Ligatures=TeX,Scale=1}
\fi
\usepackage{lmodern}
\ifPDFTeX\else  
    % xetex/luatex font selection
\fi
% Use upquote if available, for straight quotes in verbatim environments
\IfFileExists{upquote.sty}{\usepackage{upquote}}{}
\IfFileExists{microtype.sty}{% use microtype if available
  \usepackage[]{microtype}
  \UseMicrotypeSet[protrusion]{basicmath} % disable protrusion for tt fonts
}{}
\makeatletter
\@ifundefined{KOMAClassName}{% if non-KOMA class
  \IfFileExists{parskip.sty}{%
    \usepackage{parskip}
  }{% else
    \setlength{\parindent}{0pt}
    \setlength{\parskip}{6pt plus 2pt minus 1pt}}
}{% if KOMA class
  \KOMAoptions{parskip=half}}
\makeatother
\usepackage{xcolor}
\setlength{\emergencystretch}{3em} % prevent overfull lines
\setcounter{secnumdepth}{-\maxdimen} % remove section numbering
% Make \paragraph and \subparagraph free-standing
\makeatletter
\ifx\paragraph\undefined\else
  \let\oldparagraph\paragraph
  \renewcommand{\paragraph}{
    \@ifstar
      \xxxParagraphStar
      \xxxParagraphNoStar
  }
  \newcommand{\xxxParagraphStar}[1]{\oldparagraph*{#1}\mbox{}}
  \newcommand{\xxxParagraphNoStar}[1]{\oldparagraph{#1}\mbox{}}
\fi
\ifx\subparagraph\undefined\else
  \let\oldsubparagraph\subparagraph
  \renewcommand{\subparagraph}{
    \@ifstar
      \xxxSubParagraphStar
      \xxxSubParagraphNoStar
  }
  \newcommand{\xxxSubParagraphStar}[1]{\oldsubparagraph*{#1}\mbox{}}
  \newcommand{\xxxSubParagraphNoStar}[1]{\oldsubparagraph{#1}\mbox{}}
\fi
\makeatother

\usepackage{color}
\usepackage{fancyvrb}
\newcommand{\VerbBar}{|}
\newcommand{\VERB}{\Verb[commandchars=\\\{\}]}
\DefineVerbatimEnvironment{Highlighting}{Verbatim}{commandchars=\\\{\}}
% Add ',fontsize=\small' for more characters per line
\usepackage{framed}
\definecolor{shadecolor}{RGB}{241,243,245}
\newenvironment{Shaded}{\begin{snugshade}}{\end{snugshade}}
\newcommand{\AlertTok}[1]{\textcolor[rgb]{0.68,0.00,0.00}{#1}}
\newcommand{\AnnotationTok}[1]{\textcolor[rgb]{0.37,0.37,0.37}{#1}}
\newcommand{\AttributeTok}[1]{\textcolor[rgb]{0.40,0.45,0.13}{#1}}
\newcommand{\BaseNTok}[1]{\textcolor[rgb]{0.68,0.00,0.00}{#1}}
\newcommand{\BuiltInTok}[1]{\textcolor[rgb]{0.00,0.23,0.31}{#1}}
\newcommand{\CharTok}[1]{\textcolor[rgb]{0.13,0.47,0.30}{#1}}
\newcommand{\CommentTok}[1]{\textcolor[rgb]{0.37,0.37,0.37}{#1}}
\newcommand{\CommentVarTok}[1]{\textcolor[rgb]{0.37,0.37,0.37}{\textit{#1}}}
\newcommand{\ConstantTok}[1]{\textcolor[rgb]{0.56,0.35,0.01}{#1}}
\newcommand{\ControlFlowTok}[1]{\textcolor[rgb]{0.00,0.23,0.31}{\textbf{#1}}}
\newcommand{\DataTypeTok}[1]{\textcolor[rgb]{0.68,0.00,0.00}{#1}}
\newcommand{\DecValTok}[1]{\textcolor[rgb]{0.68,0.00,0.00}{#1}}
\newcommand{\DocumentationTok}[1]{\textcolor[rgb]{0.37,0.37,0.37}{\textit{#1}}}
\newcommand{\ErrorTok}[1]{\textcolor[rgb]{0.68,0.00,0.00}{#1}}
\newcommand{\ExtensionTok}[1]{\textcolor[rgb]{0.00,0.23,0.31}{#1}}
\newcommand{\FloatTok}[1]{\textcolor[rgb]{0.68,0.00,0.00}{#1}}
\newcommand{\FunctionTok}[1]{\textcolor[rgb]{0.28,0.35,0.67}{#1}}
\newcommand{\ImportTok}[1]{\textcolor[rgb]{0.00,0.46,0.62}{#1}}
\newcommand{\InformationTok}[1]{\textcolor[rgb]{0.37,0.37,0.37}{#1}}
\newcommand{\KeywordTok}[1]{\textcolor[rgb]{0.00,0.23,0.31}{\textbf{#1}}}
\newcommand{\NormalTok}[1]{\textcolor[rgb]{0.00,0.23,0.31}{#1}}
\newcommand{\OperatorTok}[1]{\textcolor[rgb]{0.37,0.37,0.37}{#1}}
\newcommand{\OtherTok}[1]{\textcolor[rgb]{0.00,0.23,0.31}{#1}}
\newcommand{\PreprocessorTok}[1]{\textcolor[rgb]{0.68,0.00,0.00}{#1}}
\newcommand{\RegionMarkerTok}[1]{\textcolor[rgb]{0.00,0.23,0.31}{#1}}
\newcommand{\SpecialCharTok}[1]{\textcolor[rgb]{0.37,0.37,0.37}{#1}}
\newcommand{\SpecialStringTok}[1]{\textcolor[rgb]{0.13,0.47,0.30}{#1}}
\newcommand{\StringTok}[1]{\textcolor[rgb]{0.13,0.47,0.30}{#1}}
\newcommand{\VariableTok}[1]{\textcolor[rgb]{0.07,0.07,0.07}{#1}}
\newcommand{\VerbatimStringTok}[1]{\textcolor[rgb]{0.13,0.47,0.30}{#1}}
\newcommand{\WarningTok}[1]{\textcolor[rgb]{0.37,0.37,0.37}{\textit{#1}}}

\providecommand{\tightlist}{%
  \setlength{\itemsep}{0pt}\setlength{\parskip}{0pt}}\usepackage{longtable,booktabs,array}
\usepackage{calc} % for calculating minipage widths
% Correct order of tables after \paragraph or \subparagraph
\usepackage{etoolbox}
\makeatletter
\patchcmd\longtable{\par}{\if@noskipsec\mbox{}\fi\par}{}{}
\makeatother
% Allow footnotes in longtable head/foot
\IfFileExists{footnotehyper.sty}{\usepackage{footnotehyper}}{\usepackage{footnote}}
\makesavenoteenv{longtable}
\usepackage{graphicx}
\makeatletter
\def\maxwidth{\ifdim\Gin@nat@width>\linewidth\linewidth\else\Gin@nat@width\fi}
\def\maxheight{\ifdim\Gin@nat@height>\textheight\textheight\else\Gin@nat@height\fi}
\makeatother
% Scale images if necessary, so that they will not overflow the page
% margins by default, and it is still possible to overwrite the defaults
% using explicit options in \includegraphics[width, height, ...]{}
\setkeys{Gin}{width=\maxwidth,height=\maxheight,keepaspectratio}
% Set default figure placement to htbp
\makeatletter
\def\fps@figure{htbp}
\makeatother

\KOMAoption{captions}{tableheading}
\makeatletter
\@ifpackageloaded{caption}{}{\usepackage{caption}}
\AtBeginDocument{%
\ifdefined\contentsname
  \renewcommand*\contentsname{Table of contents}
\else
  \newcommand\contentsname{Table of contents}
\fi
\ifdefined\listfigurename
  \renewcommand*\listfigurename{List of Figures}
\else
  \newcommand\listfigurename{List of Figures}
\fi
\ifdefined\listtablename
  \renewcommand*\listtablename{List of Tables}
\else
  \newcommand\listtablename{List of Tables}
\fi
\ifdefined\figurename
  \renewcommand*\figurename{Figure}
\else
  \newcommand\figurename{Figure}
\fi
\ifdefined\tablename
  \renewcommand*\tablename{Table}
\else
  \newcommand\tablename{Table}
\fi
}
\@ifpackageloaded{float}{}{\usepackage{float}}
\floatstyle{ruled}
\@ifundefined{c@chapter}{\newfloat{codelisting}{h}{lop}}{\newfloat{codelisting}{h}{lop}[chapter]}
\floatname{codelisting}{Listing}
\newcommand*\listoflistings{\listof{codelisting}{List of Listings}}
\makeatother
\makeatletter
\makeatother
\makeatletter
\@ifpackageloaded{caption}{}{\usepackage{caption}}
\@ifpackageloaded{subcaption}{}{\usepackage{subcaption}}
\makeatother

\ifLuaTeX
  \usepackage{selnolig}  % disable illegal ligatures
\fi
\usepackage{bookmark}

\IfFileExists{xurl.sty}{\usepackage{xurl}}{} % add URL line breaks if available
\urlstyle{same} % disable monospaced font for URLs
\hypersetup{
  pdftitle={Replication of Study What makes words special? Words as unmotivated cues (2015, Cognition)},
  pdfauthor={Jenna Brooks(j8brooks@ucsd.edu)},
  colorlinks=true,
  linkcolor={blue},
  filecolor={Maroon},
  citecolor={Blue},
  urlcolor={Blue},
  pdfcreator={LaTeX via pandoc}}


\title{Replication of Study What makes words special? Words as
unmotivated cues (2015, Cognition)}
\author{Jenna Brooks(j8brooks@ucsd.edu)}
\date{2024-11-26}

\begin{document}
\maketitle

\renewcommand*\contentsname{Table of contents}
{
\hypersetup{linkcolor=}
\setcounter{tocdepth}{3}
\tableofcontents
}

\subsection{Introduction}\label{introduction}

This study aimed to explore why verbal labels, such as the words ``dog''
or ``guitar,'' activate conceptual knowledge more effectively than
environmental sounds associated with these objects, such as the bark of
a dog or the strum of a guitar. I chose this topic because it intersects
with my interests in language learning and auditory perception. This
study finds that verbal labels (or words) are more effective than sounds
in activating abstract category concepts because labels act as
``unmotivated cues,'' broadly representing a category without specific
reference to particular instances. In contrast, sounds are ``motivated
cues'' that link directly to specific sources or instances, limiting
their effectiveness in promoting conceptual abstraction. This difference
is highlighted by experiments showing that words activate category-level
knowledge more selectively than environmental sounds.

In this experiment, participants will be presented with either a verbal
representation or environmental sound for the following categories:
bird, dog, drum, guitar, motorcycle, and phone. Participants are
presented with an auditory cue (either a word or sound) and a picture
presented 1 second after the auditory input is made. Participants are
tested on how quickly and accurately they can determine if the picture
presented matches the auditory cue they received. They will use a yes or
no button on the computer screen. Potential challenges of this study
could be sound quality of the environmental sounds to ensure they are
clearly recognizable. Additionally, finding a diverse group of
participants for this study could be a challenge.

\subsection{Methods}\label{methods}

\subsubsection{Power Analysis}\label{power-analysis}

Based on guidance from instructional staff, the sample size was
determined with an a priori power analysis with the package simr, and is
adequate to achieve at least 80\% power for detecting the effect
reported in the original study at a significance criterion of alpha =
.05 (any random effects not specified in the original paper were taken
from a small pilot study).

\subsubsection{Planned Sample}\label{planned-sample}

We plan to have a sample size of n = 50. For pre-screening, participants
must speak English fluently to ensure comprehension in the task.
Participants are recruited and compensated on the Prolific online
platform.

\subsubsection{Materials}\label{materials}

The materials were followed precisely as follows. All materials were
provided by the original authors.

``The auditory cues comprised basic-level category labels and
environmental sounds for six categories: bird, dog, drum, guitar,
motorcycle, and phone. For each category, we obtained two distinct
environmental sound cues, e.g., , , and two separate images for each
subordinate cate- gory, e.g., two electric guitars for , two acoustic
guitars for . To control for cue variability, we also used two versions
of each spoken category label: one pronounced by a female speaker, one
by a male speaker. All auditory cues were equated in duration (600 ms.)
and normalized in volume. The images were color photographs (four images
per category). The materials, obtained from online repositories, are
available for download at http://sapir.psych.wisc.edu/stimuli/
MotivatedCuesExp1A-1B.zip''(Edmiston \& Lupyan, 2015).

The link to our online experiment can be found here:
https://ucsd-psych201a.github.io/edmiston2015/

\subsubsection{Procedure}\label{procedure}

We aim to follow this procedure as closely as possible:

``On each trial participants heard a cue and saw a picture. We
instructed participants to decide as quickly and accurately as possible
if the picture they saw came from the same basic-level cate- gory as the
word or sound they heard Participants were tested in individual rooms
sitting approximately 2400 from a monitor such that images subtended 10 
10°. Trials began with a 250 ms. fixation cross followed immediately by
the auditory cue, delivered via headphones. The target image appeared
centrally 1 s after the off- set of the auditory cue and remained
visible until a response was made. Each participant completed 6 practice
and 384 test trials. If the picture matched the auditory cue (50\% of
trials) participants were instructed to respond `Yes' on a gaming
controller (e.g., or ``phone'\,' followed by a picture of any phone).
Otherwise, they were to press `No' (e.g., or ``phone'\,' followed by a
dog). All factors (cue type, congruence) var- ied randomly within
subjects. Auditory feedback (buzz or bleep) was given after each
trial''(Edmiston \& Lupyan, 2015).

We aim to follow the original procedure of Experiment 1A as precisely as
possible. However, instead of running the trials in-person, the
experiment will be conducted online using jsPsych. For this reason, the
task will be slightly different such that participants will respond to
trials using their keyboard keys instead of a gaming controller. We will
also encourage participants to wear headphones, be in a quiet area for
the auditory cues during the experiment, and provide an initial audio
check to ensure that participants have access to all stimuli presented
throughout the experiment.

\subsubsection{Analysis Plan}\label{analysis-plan}

\textbf{Clarify key analysis of interest here}\\
From the original study:

``All participants performed very accurately on all items (M = 97\%).
Response times (RTs) shorter than 250 ms. or longer than 1500 ms. were
removed (292 trials removed, 1.77\% of total).

We fit RTs for correct responses on matching trials (`Yes' responses)
with linear mixed regression using maximum likelihood estimation (Bates,
Maechler, Bolker, \& Walker, 2013), including random intercepts and
random slopes for within-subject factors and random intercepts for
repeated items (unique trial types) following the recommendations of
Barr, Levy, Scheepers, and Tily (2013). Reported below are the parameter
estimates (b) and confidence intervals for each contrast of interest.
Significance tests were calculated using chi-square tests that compared
nested models---models with and without the factor of interest---on
improvement in log-likelihood.''

For our replication, we will also remove trials where response times are
shorter than 250 ms or longer than 1500ms from the analysis. We will
then run a similar linear mixed regression model, as was performed the
original study. We will also use chi-square tests to assess significance
of the results. We anticipate that a successful replication of the
original study will yield similar effects. More specifically, we should
find that verbal labels elicit the lowest overall reaction times, and
that congruent sounds elicit lower reaction times than incongruent
sounds.

\subsubsection{Differences from Original
Study}\label{differences-from-original-study}

This replication will differ from the original study in both setting and
procedure. Instead of being conducted in a lab, the experiments will
take place online using Prolific. Variations in participants' device
performance and internet speed may influence reaction times.
Additionally, participants will respond to trials using a keyboard
rather than the gaming controller used previously. Despite these
changes, we expect these differences will have minimal impact on the
results or the ability to replicate the effect reported in the original
study.

\subsection{Pilot A}\label{pilot-a}

For our Pilot A (with non-naive participants), we aimed to replicate the
experimental paradigm described in the study. We collected an initial
sample of 3 participants from friends. The GitHub page for our paradigm
is linked below:
\href{https://ucsd-psych201a.github.io/edmiston2015/}{Link to
experimental paradigm}

This data was imported and uploaded to the data folder in the project
repository, which can be found on our GitHub repository here:

https://github.com/ucsd-psych201a/edmiston2015/tree/main/data

\subsection{Pilot B}\label{pilot-b}

For our Pilot B (with naive participants), we collected data from five
participants on Prolific. The following ``Results'' section (Data
Preparation and Confirmatory Analysis) is run on Pilot B data for the
time-being. As predicted, the participants took an average of about 25
minutes to complete the experiment.

\href{https://github.com/ucsd-psych201a/edmiston2015/tree/main/data/pilot_b}{Link
to Pilot B data}

\subsection{Pre-Registration}\label{pre-registration}

Our Pre-Registration can be found here:
https://github.com/ucsd-psych201a/edmiston2015/tree/main/prereg

\subsubsection{Methods Addendum (Post Data
Collection)}\label{methods-addendum-post-data-collection}

You can comment this section out prior to final report with data
collection.

\paragraph{Actual Sample}\label{actual-sample}

Sample size, demographics, data exclusions based on rules spelled out in
analysis plan

\paragraph{Differences from pre-data collection methods
plan}\label{differences-from-pre-data-collection-methods-plan}

Any differences from what was described as the original plan, or
``none''.

\subsection{Results}\label{results}

\subsubsection{Data preparation}\label{data-preparation}

We are replicating Experiment 1A from the paper. To clean the data we
will remove incorrect responses and practice trials. As for data
exclusion rules, we will follow the parameters set by the original study
in which ``Response times (RTs) shorter than 250 ms. or longer than 1500
ms. {[}will be{]} removed''. This accounts for outliers that may skew
results. This data preparation also labels the conditions for later
analysis (i.e.~\texttt{congruent}, \texttt{incogruent}).

\begin{Shaded}
\begin{Highlighting}[]
\DocumentationTok{\#\#\# Data Preparation}

\DocumentationTok{\#\#\#\# Load Relevant Libraries and Functions}
\FunctionTok{library}\NormalTok{(tidyverse)}
\end{Highlighting}
\end{Shaded}

\begin{verbatim}
-- Attaching core tidyverse packages ------------------------ tidyverse 2.0.0 --
v dplyr     1.1.4     v readr     2.1.5
v forcats   1.0.0     v stringr   1.5.1
v ggplot2   3.5.1     v tibble    3.2.1
v lubridate 1.9.3     v tidyr     1.3.1
v purrr     1.0.2     
-- Conflicts ------------------------------------------ tidyverse_conflicts() --
x dplyr::filter() masks stats::filter()
x dplyr::lag()    masks stats::lag()
i Use the conflicted package (<http://conflicted.r-lib.org/>) to force all conflicts to become errors
\end{verbatim}

\begin{Shaded}
\begin{Highlighting}[]
\FunctionTok{library}\NormalTok{(readr)}
\FunctionTok{library}\NormalTok{(lme4)}
\end{Highlighting}
\end{Shaded}

\begin{verbatim}
Loading required package: Matrix

Attaching package: 'Matrix'

The following objects are masked from 'package:tidyr':

    expand, pack, unpack
\end{verbatim}

\begin{Shaded}
\begin{Highlighting}[]
\DocumentationTok{\#\#\#\# Import data}
\CommentTok{\#replace this with your own path (for now)}
\NormalTok{folder\_path }\OtherTok{\textless{}{-}} \StringTok{"../data/pilot\_b"}
\NormalTok{csv\_files }\OtherTok{\textless{}{-}} \FunctionTok{list.files}\NormalTok{(folder\_path, }\AttributeTok{pattern =} \StringTok{"*.csv"}\NormalTok{, }\AttributeTok{full.names =} \ConstantTok{TRUE}\NormalTok{)}
\NormalTok{df\_list }\OtherTok{\textless{}{-}} \FunctionTok{lapply}\NormalTok{(csv\_files, read\_csv)}
\end{Highlighting}
\end{Shaded}

\begin{verbatim}
Rows: 390 Columns: 15
-- Column specification --------------------------------------------------------
Delimiter: ","
chr (12): correct_response, response, img_fname, img_category, img_subtype, ...
dbl  (3): rt, time_elapsed, img_version

i Use `spec()` to retrieve the full column specification for this data.
i Specify the column types or set `show_col_types = FALSE` to quiet this message.
Rows: 390 Columns: 15
-- Column specification --------------------------------------------------------
Delimiter: ","
chr (12): correct_response, response, img_fname, img_category, img_subtype, ...
dbl  (3): rt, time_elapsed, img_version

i Use `spec()` to retrieve the full column specification for this data.
i Specify the column types or set `show_col_types = FALSE` to quiet this message.
Rows: 390 Columns: 15
-- Column specification --------------------------------------------------------
Delimiter: ","
chr (12): correct_response, response, img_fname, img_category, img_subtype, ...
dbl  (3): rt, time_elapsed, img_version

i Use `spec()` to retrieve the full column specification for this data.
i Specify the column types or set `show_col_types = FALSE` to quiet this message.
Rows: 390 Columns: 15
-- Column specification --------------------------------------------------------
Delimiter: ","
chr (12): correct_response, response, img_fname, img_category, img_subtype, ...
dbl  (3): rt, time_elapsed, img_version

i Use `spec()` to retrieve the full column specification for this data.
i Specify the column types or set `show_col_types = FALSE` to quiet this message.
Rows: 390 Columns: 15
-- Column specification --------------------------------------------------------
Delimiter: ","
chr (12): correct_response, response, img_fname, img_category, img_subtype, ...
dbl  (3): rt, time_elapsed, img_version

i Use `spec()` to retrieve the full column specification for this data.
i Specify the column types or set `show_col_types = FALSE` to quiet this message.
\end{verbatim}

\begin{Shaded}
\begin{Highlighting}[]
\NormalTok{combined\_df }\OtherTok{\textless{}{-}} \FunctionTok{bind\_rows}\NormalTok{(df\_list)}

\DocumentationTok{\#\#\#\# Data exclusion / filtering}
\CommentTok{\#exclude \textquotesingle{}No\textquotesingle{} trials }
\NormalTok{combined\_df }\OtherTok{\textless{}{-}}\NormalTok{ combined\_df }\SpecialCharTok{\%\textgreater{}\%}
  \FunctionTok{mutate}\NormalTok{(}\AttributeTok{response =} \FunctionTok{as.character}\NormalTok{(response)) }\SpecialCharTok{\%\textgreater{}\%}
  \FunctionTok{filter}\NormalTok{(response }\SpecialCharTok{==} \StringTok{"f"}\NormalTok{)}
\CommentTok{\#exclude practice trials }
\NormalTok{combined\_df }\OtherTok{\textless{}{-}}\NormalTok{ combined\_df }\SpecialCharTok{\%\textgreater{}\%}
  \FunctionTok{filter}\NormalTok{(exp\_part }\SpecialCharTok{==} \StringTok{"actual"}\NormalTok{)}

\CommentTok{\#rename "sound\_subtype" to "cue"}
\NormalTok{combined\_df }\OtherTok{\textless{}{-}}\NormalTok{ combined\_df }\SpecialCharTok{\%\textgreater{}\%}
  \FunctionTok{rename}\NormalTok{(}\AttributeTok{cue =}\NormalTok{ sound\_subtype)}

\CommentTok{\#create congruency column }
\NormalTok{combined\_df }\OtherTok{\textless{}{-}}\NormalTok{ combined\_df }\SpecialCharTok{\%\textgreater{}\%}
  \FunctionTok{mutate}\NormalTok{(}\AttributeTok{congruency =} \FunctionTok{case\_when}\NormalTok{(}
\NormalTok{    cue }\SpecialCharTok{==} \StringTok{"label"} \SpecialCharTok{\textasciitilde{}} \StringTok{"label"}\NormalTok{,}
\NormalTok{    img\_subtype }\SpecialCharTok{\%in\%} \FunctionTok{c}\NormalTok{(}\StringTok{"song"}\NormalTok{, }\StringTok{"york"}\NormalTok{, }\StringTok{"bongo"}\NormalTok{, }\StringTok{"acoustic"}\NormalTok{, }\StringTok{"harley"}\NormalTok{, }\StringTok{"rotary"}\NormalTok{) }\SpecialCharTok{\&}\NormalTok{ sound\_version }\SpecialCharTok{==} \StringTok{"A"} \SpecialCharTok{\textasciitilde{}} \StringTok{"incongruent"}\NormalTok{,}
    \ConstantTok{TRUE} \SpecialCharTok{\textasciitilde{}} \StringTok{"congruent"}
\NormalTok{  ))}


\CommentTok{\#filter reaction time }
\NormalTok{combined\_df }\OtherTok{\textless{}{-}}\NormalTok{ combined\_df }\SpecialCharTok{\%\textgreater{}\%}
  \FunctionTok{filter}\NormalTok{(rt }\SpecialCharTok{\textgreater{}}\DecValTok{250}\NormalTok{, rt }\SpecialCharTok{\textless{}}\DecValTok{1500}\NormalTok{)}

\DocumentationTok{\#\#\#\# Prepare data for analysis {-} create columns etc.}
\NormalTok{combined\_df }\OtherTok{\textless{}{-}}\NormalTok{ combined\_df }\SpecialCharTok{\%\textgreater{}\%}
  \FunctionTok{select}\NormalTok{(rt, ID, sound\_category,cue, congruency)}
\end{Highlighting}
\end{Shaded}

\subsection{Design Overview}\label{design-overview}

Within-subjects design with 2 factors: Auditory Cue (Environmental Sound
vs.~Label) and Match to Basic Category (Match vs.~No Match). Congruency
is further manipulated within the Matching Environmental Sound
condition.

Reaction time was the only measure taken.

It uses within-participants design where everyone does each condition.

Measures were repeated for 384 experimental trials.

They didn't take any measures to reduce demand characteristics.

To improve the experiment attention checks could be added. Given that
incorrect responses could be labeled incongruent, incorrect responses
could have also been included in the data. This sample could not be
representative of all populations because it only studied undergraduates
and WEIRD populations.

\subsubsection{Confirmatory analysis}\label{confirmatory-analysis}

This analysis aims to determine whether (1) verbal labels lead to faster
response times compared to sound labels, and (2) congruent sounds
produce faster responses than incongruent sounds, in line with the
findings of Experiment 1A from the original study.

As outlined in our analysis plan, we adopt the approach used by Edmiston
\& Lupyan (2015), modeling reaction times for correct responses in
matching trials across three cue conditions (verbal label, congruent
sound, incongruent sound) using a linear mixed-effects regression model.
The model incorporates random intercepts and slopes for within-subject
factors, as well as random intercepts for repeated measures (unique
trial types). The primary variable of interest is the ``condition,''
representing whether the trial used a label, a congruent sound, or an
incongruent sound.

In the next section, we will present parameter estimates and confidence
intervals for each contrast of interest. These estimates indicate the
extent to which each condition affects reaction times. Additionally, to
assess how strongly the factors of interest explain the observed
patterns, we conduct chi-square tests to compare nested models---those
with and without the factor of interest---based on improvements in
log-likelihood. The resulting p-values will indicate the statistical
significance of the effects.

We anticipate that the model outputs will align closely with the results
reported by Edmiston \& Lupyan (2015). More specifically, incongruent
environmental sounds will elicit longer response times in comparison to
congruent environmental sounds. Verbal labels will elicit the shortest
response time.

\emph{Side-by-side graph with original graph is ideal here}

\begin{Shaded}
\begin{Highlighting}[]
\FunctionTok{library}\NormalTok{(lmerTest)}
\end{Highlighting}
\end{Shaded}

\begin{verbatim}

Attaching package: 'lmerTest'
\end{verbatim}

\begin{verbatim}
The following object is masked from 'package:lme4':

    lmer
\end{verbatim}

\begin{verbatim}
The following object is masked from 'package:stats':

    step
\end{verbatim}

\begin{Shaded}
\begin{Highlighting}[]
\FunctionTok{library}\NormalTok{(emmeans)}
\end{Highlighting}
\end{Shaded}

\begin{verbatim}
Welcome to emmeans.
Caution: You lose important information if you filter this package's results.
See '? untidy'
\end{verbatim}

\begin{Shaded}
\begin{Highlighting}[]
\FunctionTok{library}\NormalTok{(lme4)}

\CommentTok{\#shows you a comparsion between reference (congruent) and the other two values (incongruent and label)}
\NormalTok{model\_full }\OtherTok{\textless{}{-}} \FunctionTok{lmer}\NormalTok{(rt }\SpecialCharTok{\textasciitilde{}}\NormalTok{ congruency }\SpecialCharTok{+}\NormalTok{ (}\DecValTok{1} \SpecialCharTok{+}\NormalTok{ congruency}\SpecialCharTok{|}\NormalTok{ID) }\SpecialCharTok{+}\NormalTok{ (}\DecValTok{1}\SpecialCharTok{|}\NormalTok{sound\_category), }\AttributeTok{data =}\NormalTok{ combined\_df)}
\end{Highlighting}
\end{Shaded}

\begin{verbatim}
boundary (singular) fit: see help('isSingular')
\end{verbatim}

\begin{Shaded}
\begin{Highlighting}[]
\FunctionTok{summary}\NormalTok{(model\_full)}
\end{Highlighting}
\end{Shaded}

\begin{verbatim}
Linear mixed model fit by REML. t-tests use Satterthwaite's method [
lmerModLmerTest]
Formula: rt ~ congruency + (1 + congruency | ID) + (1 | sound_category)
   Data: combined_df

REML criterion at convergence: 11919.7

Scaled residuals: 
    Min      1Q  Median      3Q     Max 
-2.1094 -0.6902 -0.1522  0.4877  3.6029 

Random effects:
 Groups         Name                  Variance Std.Dev. Corr       
 sound_category (Intercept)             144.0   12.00              
 ID             (Intercept)           44075.9  209.94              
                congruencyincongruent   787.7   28.07   -0.50      
                congruencylabel        2577.0   50.76   -0.74  0.95
 Residual                             42596.0  206.39              
Number of obs: 883, groups:  sound_category, 6; ID, 5

Fixed effects:
                      Estimate Std. Error      df t value Pr(>|t|)   
(Intercept)            718.066     94.721   4.025   7.581  0.00158 **
congruencyincongruent  -16.156     25.975   8.118  -0.622  0.55101   
congruencylabel        -37.768     27.266   3.877  -1.385  0.24037   
---
Signif. codes:  0 '***' 0.001 '**' 0.01 '*' 0.05 '.' 0.1 ' ' 1

Correlation of Fixed Effects:
            (Intr) cngrncyn
cngrncyncng -0.293         
congrncylbl -0.660  0.572  
optimizer (nloptwrap) convergence code: 0 (OK)
boundary (singular) fit: see help('isSingular')
\end{verbatim}

\begin{Shaded}
\begin{Highlighting}[]
\CommentTok{\#get 95\% CI}
\FunctionTok{confint.merMod}\NormalTok{(model\_full,}\AttributeTok{method=}\StringTok{"Wald"}\NormalTok{)}
\end{Highlighting}
\end{Shaded}

\begin{verbatim}
                          2.5 %    97.5 %
.sig01                       NA        NA
.sig02                       NA        NA
.sig03                       NA        NA
.sig04                       NA        NA
.sig05                       NA        NA
.sig06                       NA        NA
.sig07                       NA        NA
.sigma                       NA        NA
(Intercept)           532.41644 903.71625
congruencyincongruent -67.06601  34.75379
congruencylabel       -91.20730  15.67183
\end{verbatim}

\begin{Shaded}
\begin{Highlighting}[]
\CommentTok{\#post{-}hoc test: allows you to examine relationship between incongruent and label}
\NormalTok{model\_full }\SpecialCharTok{\%\textgreater{}\%} 
  \FunctionTok{emmeans}\NormalTok{(pairwise }\SpecialCharTok{\textasciitilde{}}\NormalTok{ congruency,}
          \CommentTok{\#adjusts p values so that it is more difficult to get a significance (correction method)}
          \AttributeTok{adjust =} \StringTok{"bonferroni"}\NormalTok{) }\SpecialCharTok{\%\textgreater{}\%} 
  \FunctionTok{pluck}\NormalTok{(}\StringTok{"contrasts"}\NormalTok{)}
\end{Highlighting}
\end{Shaded}

\begin{verbatim}
Cannot use mode = "kenward-roger" because *pbkrtest* package is not installed
\end{verbatim}

\begin{verbatim}
 contrast                estimate   SE   df t.ratio p.value
 congruent - incongruent     16.2 26.0 8.12   0.622  1.0000
 congruent - label           37.8 27.3 3.88   1.385  0.7211
 incongruent - label         21.6 24.7 9.62   0.876  1.0000

Degrees-of-freedom method: satterthwaite 
P value adjustment: bonferroni method for 3 tests 
\end{verbatim}

\begin{Shaded}
\begin{Highlighting}[]
\CommentTok{\#model shows us that incongruent takes longer than congruent b = 28.7, that label is shorter than congruent (b = {-}15.52), and that incongruent takes longer than label b = 44.3 (find 95\% CI) }
\end{Highlighting}
\end{Shaded}

\emph{Side-by-side graph with original graph is ideal here}

\subsubsection{Exploratory analyses}\label{exploratory-analyses}

Any follow-up analyses desired (not required).

\subsection{Discussion}\label{discussion}

\subsubsection{Summary of Replication
Attempt}\label{summary-of-replication-attempt}

Open the discussion section with a paragraph summarizing the primary
result from the confirmatory analysis and the assessment of whether it
replicated, partially replicated, or failed to replicate the original
result.

\subsubsection{Commentary}\label{commentary}

Add open-ended commentary (if any) reflecting (a) insights from
follow-up exploratory analysis, (b) assessment of the meaning of the
replication (or not) - e.g., for a failure to replicate, are the
differences between original and present study ones that definitely,
plausibly, or are unlikely to have been moderators of the result, and
(c) discussion of any objections or challenges raised by the current and
original authors about the replication attempt. None of these need to be
long.




\end{document}
